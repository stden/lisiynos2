% Author: Ivan Kazmenko
% Text: Ivan Kazmenko
% Origin: 20070209, Lisiy Nos, Simple Graph Training
\begin{problem}{Связность}
{connect.in}{connect.out}
{2 секунды}{256 мебибайт}{}

 В этой задаче требуется проверить, что граф является {\it связным}, то есть
 что из любой вершины можно по рёбрам этого графа попасть в любую другую.

\InputFile

 В первой строке входного файла заданы числа $N$ и $M$ через пробел "---
 количество вершин и рёбер в графе, соответственно
 ($1 \leqslant N \leqslant 100$, $0 \leqslant M \leqslant 10 \, 000$).
 Следующие $M$ строк содержат по два числа $u_i$ и $v_i$ через пробел
 ($1 \leqslant u_i, \, v_i \leqslant N$);
 каждая такая строка означает, что в графе существует ребро между вершинами
 $u_i$ и $v_i$.

\OutputFile

 Выведите ``\t{YES}'', если граф является связным, и ``\t{NO}'' в противном
 случае.

\Examples

\begin{example}
\exmp{
3 2
1 2
3 2
}{
YES
}%
\exmp{
3 1
1 3
}{
NO
}%
\end{example}

\end{problem}
